\documentclass[10pt,slidestop,mathserif]{beamer}
%[slidestop] puts frame titles & contents on the top left corner (default = [slidescentered])
%[red] changes navigation bars and title to reddish color
%[mathserif] use serif fonts for representing formulas instead of sans serif (default = mathsans)
%[notes] adds notes to PDF screen
%[compress] the navigationbar in one line
%[notesonly] make only notes

%\mode<presentation>
%\mode<article>

\usepackage{xeCJK}
\usepackage{hyperref}
%\usepackage{amsmath} %for math AMS fonts
%\usepackage{graphicx} %to include figures
%\usepackage{subfigure} %to have figures in figures
%\usepackage{multimedia} %to include movies
\usepackage{tikz} %to draw diagram


%Fonts
\setCJKmainfont[BoldFont={Adobe Heiti Std}, ItalicFont={Adobe Kaiti Std}]{Adobe Song Std}
\setCJKmonofont{Adobe Fangsong Std}
\setmainfont[Mapping=tex-text]{Liberation Serif}
\setsansfont{Liberation Sans}
\setmonofont{Bitstream Vera Sans Mono}
\punctstyle{kaiming}
%\setCJKfamilyfont{kai}{KaiTi}
%\setCJKfamilyfont{hei}{SimHei}
%\setCJKmainfont{SimSun}




%NewCommands
\newcommand\abs[1]{\left\lvert #1 \right\rvert}
\newcommand\floor[1]{\left\lfloor #1 \right\rfloor}
\newcommand\ceil[1]{\left\lceil #1 \right\rceil}
\newcommand\yin[1]{\textit{#1}}
\newcommand\yang[1]{\textbf{#1}}
%\newcommand{\kai}{\CJKfamily{kai}}
%\newcommand{\hei}{\CJKfamily{hei}}
%\newcommand\TT{\rule{0pt}{2.6ex}}
%\newcommand\BB{\rule[-1.2ex]{0pt}{0pt}}



%PageStyle
\setlength{\paperheight}{845pt}
\setlength{\paperwidth}{597pt}
\setlength{\parindent}{0em}
\setlength{\textheight}{663pt}
\setlength{\textwidth}{424pt}
\setlength{\oddsidemargin}{18pt}
\setlength{\evensidemargin}{18pt}
\setlength{\headheight}{12pt}
\setlength{\topmargin}{0pt}
\setlength{\marginparsep}{11pt}
\setlength{\footskip}{30pt}
\setlength{\hoffset}{0pt}
\setlength{\headsep}{25pt}
\setlength{\marginparwidth}{54pt}
\setlength{\voffset}{0pt}
%\pagestyle{fancy}
%\fancyhead{}
%\fancyfoot{} % clear all fields
%\fancyfoot[LF]{Copyright by Yinyanghu}
%\fancyfoot[RF]{\thepage}
%\renewcommand{\footrulewidth}{0.4pt}
%\renewcommand{\headrulewidth}{0.0pt}


%Enumerate
\renewcommand{\labelenumi}{\bfseries{\arabic{enumi}.}}
\renewcommand{\labelenumii}{\bfseries{\alph{enumii}.}}



\begin{document}

%%Fix the bug of guide button; Beamer 3.17+ fixed!
\makeatletter
\def\beamer@linkspace#1{%
	\begin{pgfpicture}{0pt}{-1.5pt}{#1}{5.5pt}
		\pgfsetfillopacity{0}
		\pgftext[x=0pt,y=-1.5pt]{.}
		\pgftext[x=#1,y=5.5pt]{.}
	\end{pgfpicture}
}
\makeatother


\title{Example}
\subtitle{Beamer}
\author{{\scriptsize\bfseries Yinyanghu} \\ {\scriptsize\bfseries 3.141...}}
\institute{NJU}
\subject{Computer Science}
\date{\today}

%\title[Crisis] % (optional, only for long titles)
%{The Economics of Financial Crisis}
%\subtitle{Evidence from India}
%\author[Author, Anders] % (optional, for multiple authors)
%{F.~Author\inst{1} \and S.~Anders\inst{2}}
%\institute[Universitäten Hier und Dort] % (optional)
%{
%	\inst{1}%
%	Institut für Informatik\\
%	Universität Hier
%	\and
%	\inst{2}%
%	Institut für theoretische Philosophie\\
%	Universität Dort
%}
%\date[KPT 2004] % (optional)
%{Konferenz über Präsentationstechniken, 2004}
%\subject{Informatik}

%[plain] for plane frame style
%[containsverbatim] for using verbatim environment and \verb command
%[allowframebreaks] for automatic split of frames if the contents do not fit in a single slide
%[shrink] for shrinking the contents to fit in a single slide
%[squeeze] for squeezing vertical space
\begin{frame}[plain]
	\titlepage
\end{frame}

\begin{frame}
	\tableofcontents
\end{frame}

\section{A}
\subsection{A1}
\begin{frame}[shrink]
	\frametitle{A1}
	\framesubtitle{test}

	Erd\H os
	\begin{block}{P}
		\begin{itemize}
			\item A %\pause
			\item B %\pause
			\item C %\pause
		\end{itemize}
	\end{block}

	\footnote{On a fast machine.}
\end{frame}

\section{B}

\subsection{B1}
\begin{frame}
	\transdissolve
	\frametitle{B1}
	\begin{columns}[t]
		\column{0.33\textwidth}
		. . . contents A . . .
		\column{0.33\textwidth}
		. . . contents B . . .
		\column{0.33\textwidth}
		. . . contents C . . .
	\end{columns}
\end{frame}

\subsection{B2}
\begin{frame}
	\frametitle{B2}
	\label{th1}
	\begin{theorem}
		\begin{equation}
			e^{i\pi} = -1
		\end{equation}
	\end{theorem}

	\begin{proof}
		\begin{equation}
			A = B + C
		\end{equation}
	\end{proof}

	\begin{example}
		Write your fantastic \\
	\end{example}

	\begin{definition}
		Write your fantastic \\
	\end{definition}

	%corollary, alertblock, fact, lemma

\end{frame}

\section{C}
\begin{frame}[containsverbatim]
	\frametitle{C1}
	\begin{verbatim}

         F()
A -----------------> B
         G()
	\end{verbatim}
\end{frame}

\begin{frame}
	\frametitle{C2}
	\hyperlink{th1}{\beamergotobutton{Jump to Theorem \#1}}
	\hypertarget{th1}{help}
\end{frame}

\begin{frame}
	\frametitle{C3}
	\begin{itemize}
		\item <+-| alert@+> Every thing
		\item <+-| alert@+> that has
		\item <+-| alert@+> beginning
		\item <+-| alert@+> has end.
	\end{itemize}
\end{frame}

\begin{frame}
	\frametitle{C4}
	\begin{itemize}
		\item<2-> \alt<2>{\color{blue}Everything}{\color{gray} Everything}
		\item<2-> \alt<3>{\color{blue}that has}{\color{gray} that has}
		\item<2-> \alt<4>{\color{blue}beginning}{\color{gray} beginning}
		\item<2-> \alt<5>{\color{blue}has end.}{\color{gray} has end.}
	\end{itemize}

\end{frame}

%Overlay
%\pause
%

\section{D}
\begin{frame}
	\frametitle{D1}
	\begin{block}{Open Questions}
		Is every even number the sum of two primes?
		\cite{Goldbach1742}
	\end{block}

\end{frame}

\begin{frame}[fragile]
	\frametitle{An Algorithm For Finding Primes Numbers.}
	\begin{verbatim}
	int main (void)
	{
		std::vector<bool> is_prime (100, true);
		for (int i = 2; i < 100; i++)
		if (is_prime[i])
		{
			std::cout << i << " ";
			for (int j = i; j < 100; is_prime [j] = false, j+=i);
		}
		return 0;
	}
	\end{verbatim}
	\begin{uncoverenv}<2>
		Note the use of \verb|std::|.
	\end{uncoverenv}
\end{frame}

%lecture in <article mode>
%\lecture{Vector Spaces}{week 1}
%\section{Introduction}
%\begin{frame}
%	\frametitle{AAA}
%\end{frame}
%\section{Summary}
%\begin{frame}
%	\frametitle{BBB}
%\end{frame}
%
%\lecture{Scalar Products}{week 2}
%\section{Introduction}
%\begin{frame}
%	\frametitle{AAA}
%\end{frame}
%\section{Summary}
%\begin{frame}
%	\frametitle{BBB}
%\end{frame}

\begin{frame}
	\frametitle{BBB}
	\begin{description}
		\item[Lion] King of the savanna.
		\item[Tiger] King of the jungle.
	\end{description}
	000\footnote[frame, 1]{Der Spiegel, 4/04, S.~90.}

	\begin{beamercolorbox}{beamer color}
		Text
	\end{beamercolorbox}
	AAA\footnote{Not proved.}

	\begin{beamerboxesrounded}[upper=block head,lower=block body,shadow=true]{Theorem}
		$A = B$.
	\end{beamerboxesrounded}
	BBB\footnote{On a fast machine.}

\end{frame}


\begin{frame}
	\frametitle{CCC}
	\begin{abstract}
		Text
	\end{abstract}

	\begin{verse}
		Text
	\end{verse}

	\begin{quotation}
		Text
	\end{quotation}

	\begin{quote}
		Text
	\end{quote}
\end{frame}

\begin{frame}
	\frametitle{Tikz}
	\begin{figure}[h]
		\begin{tikzpicture}
		%color one half of a unit circle
			\begin{scope}
				\clip (0,0) circle (1cm);
				\fill[black] (0cm,1cm) rectangle (-1cm, -1cm);
			\end{scope}

		%fill heads
			\fill[black] (0,0.5) circle (0.5cm);
			\fill[white] (0,-0.5) circle (0.5cm);

		%fill eyes
			\fill[white] (0,0.5) circle (0.1cm);
			\fill[black] (0,-0.5) circle (0.1cm);

		%outer line
			\draw (0,0) circle (1cm);

		\end{tikzpicture}
	\end{figure}
\end{frame}


\begin{frame}
	\frametitle{Deterministic Finite Automaton}
\end{frame}

\begin{thebibliography}{10}
	\bibitem{Goldbach1742}[Goldbach, 1742]
		Christian Goldbach.
		\newblock A problem we should try to solve before the ISPN ’43 deadline,
		\newblock \emph{Letter to Leonhard Euler}, 1742.
\end{thebibliography}

\end{document}
